%----------------------------------------------------------------
%
%  File    :  chapter2.tex
%
%  Authors :  David Lechner, FH Campus Wien, Austria
% 
%  Created :  10 Oct 2019
% 
%  Changed :  10 Oct 2019
% 
%----------------------------------------------------------------


\chapter{Hintergrund}
\label{chap:back}

Textkörper mit noch einem Bild

\begin{figure}[htbp]
	\centering
		\includegraphics{images/birne}
	\caption{Eine Glühbirne}
	\label{fig:birne}
\end{figure}



\section{Unterkapitel 21}
\label{sec:Unterkapitel21}

Textkörper mit Tabelle.

\begin{table*}[htbp]
	\centering
		\begin{tabular}{|l|c|r|}
		\hline
		\rowcolor[gray]{0.9}
		Spalte 1 & Spalte 2 & Spalte 3 \\
		\hline
		Affen & Giraffen & Löwen \\
		äpfel & Birnen & Bananen \\
		Irgend & et & was \\
		\hline	
		\end{tabular}
	\caption{Beispiel für eine Tabelle}
	\label{tab:BeispielFuerEineTabelle}
\end{table*}

Man beachte die Gegenüberstellung in Tabelle \ref{tab:BeispielFuerEineTabelle}.

\section{Unterkapitel 23}
\label{sec:Unterkapite23}

Aufzählungen:

Nummeriert:

\begin{enumerate}
	\item Punkt 1
	\item Punkt 2
\end{enumerate}

Mit Bullet Points:

\begin{itemize}
	\item Punkt 1
	\item Punkt 2
\end{itemize}

Mit Beschreibungen:

\begin{description}
	\item[Item 1] das ist der 1.Punkt
	\item[Item 2] und das der 2.
\end{description}


Auch Programmcodes können an entsprechender Stelle eingefügt werden, man beachte dazu auch Listing \ref{lst:conv}.

% see also http://mirror.easyname.at/ctan/macros/latex/contrib/listings/listings.pdf for options

\begin{lstlisting}[frame=lines, caption=Simple Listing, captionpos=b, label = lst:conv, language=C, showstringspaces=false]
#include <stdio.h>
int main()
{
	int i, n, t1 = 0, t2 = 1, nextTerm;

	printf("Enter the number of terms: ");
	scanf("%d", &n);

	printf("Fibonacci Series: ");

	for (i = 1; i <= n; ++i)
	{
		printf("%d, ", t1);
		nextTerm = t1 + t2;
		t1 = t2;
		t2 = nextTerm;
	}
	return 0;
}
\end{lstlisting}

Und zuguterletzt, Formeln mitten im Fliesstext, wie z.B. $a^2+b^2=c^2$, in einem Absatz.
