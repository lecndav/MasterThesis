%----------------------------------------------------------------
%
%  File    :  chapter1.tex
%
%  Authors :  David Lechner, FH Campus Wien, Austria
% 
%  Created :  10 Oct 2019
% 
%  Changed :  10 Oct 2019
% 
%----------------------------------------------------------------


\chapter{Einführung}
\label{chap:intro}

Textkörper mit Bild

\begin{figure}[htbp]
	\centering
		\includegraphics[height=5cm]{images/buecher}
	\caption{Ein Stapel Bücher}
	\label{fig:buecher}
\end{figure}


Textkörper Fortsetzung mit Verweis auf den wundervollen Stapel Bücher in Abbildung \ref{fig:buecher}. 


\section{Unterkapitel 1}
\label{sec:Unterkapitel1}

Textkörper mit Formel:

\begin{equation}
U(j\omega)=\int^{\infty}_{-\infty}{u(t) \cdot e^{-j\omega t}dt}
\label{form:form1}
\end{equation}

Textkörper Fortsetzung mit Verweis auf Formel \ref{form:form1}. Und nicht zu vergessen: es gibt auch noch eine tolle Abbildung in Kapitel \ref{chap:intro}, nämlich Abbildung \ref{fig:buecher}. 

\subsection{Unter-Unterkapitel11}
\label{sec:UnterUnterkapitel11}

Textkörper mit Fussnote\footnote{Fussnotentext}.

\subsection{Unter-Unterkapitel 12}
\label{sec:UnterUnterkapitel12}

Textkörper mit direktem Zitat:
``Repeated application of the group operations leads to the definition of the scalar multiplication.'' \cite{Koschuch2006}

\subsection{Unter-Unterkapitel 13}
\label{sec:UnterUnterkapitel13}

Textkörper mit indirektem Zitat.
Der Theorie nach sollten Berechnungen über Primkörpern auf einem handelsüblichen Prozessor signifikant schneller sein als über einem allgemeinen binären Erweiterungskörper. \cite{Hankerson2004}