%----------------------------------------------------------------
%
%  File    :  introduction.tex
%
%  Authors :  David Lechner, FH Campus Wien, Austria
%
%  Created :  10 Oct 2019
%
%  Changed :  10 Oct 2019
%
%----------------------------------------------------------------


\chapter{Einleitung}
\label{chap:intro}

In der heutigen Zeit werden Unmengen an Daten von verschiedensten Geräten generiert und versendet. Den Anfang haben Smartphones, dann Wearables gemacht. Mit dem Aufkommen des Bereichs Internet of Things (IoT) kann jegliches technische Gerät – von der Lampe bis hin zur Fertigungsmaschine – mit dem Internet verbunden sein und Status- beziehungsweise Messdaten übermitteln. Dieser Trend macht auch vor Fahrzeugen keinen Halt. Schon längst sind moderne Autos mit LTE-, GPS und Wifi-Modulen ausgestattet und senden Daten unter anderem zum Hersteller. Gartner prognostiziert für das Jahr 2020 470 Millionen vernetzte Fahrzeuge \cite{Gartner2019}. In Zukunft werden wahrscheinlich alle Fahrzeuge mit etlichen Sensoren ausgestattet sein und kommunizieren untereinander, mit der Umwelt, dem Fahrer oder sonstigen Service-Anbieter. Dies gründet vor allem auf den wachsenden Themen Vehicle-to-Everything (V2X) und Autonomes Fahren. Insbesondere beim Letztgenannten macht die generierte Datengröße noch einen großen Sprung, da neben Sensoren auch Radar und Videokameras zur Umwelterkennung hinzukommen. Jedoch versenden schon heute Electronic Control Units (ECUs) Daten im Auto, wie zum Beispiel Lenkradwinkel, Gangposition und Bremsdruck, welche für Sicherheits- und Komfortfunktionen genutzt werden. Aus diesen Daten können viele Informationen gewonnen werden. Eine davon ist, den Lenker eines Autos während der Fahrt nur durch das Fahrverhalten zu identifizieren.

Daraus lassen sich weitere unterschiedliche Anwendungen ableiten. Einige von ihnen schaffen Komfort und erleichtern in gewisser Weise das Leben des Fahrers. Andere indes könnten gegen den Fahrzeughalter und der Fahrerin selbst eingesetzt werden, diese gehen mit datenschutzrechtlichen Bedenken einher.

Doch zunächst zu den Anwendungen, welche eine positive Auswirkung haben können. Moderne Autos – vor allem jene mit einem Automatik Getriebe – bieten die Möglichkeit, sich an den Fahrstil anzupassen. Wenn beispielsweise eine Person zum schnelleren Beschleunigen neigt, lernt dies das Auto und schaltet demnach erst bei einer höheren Motordrehzahl in den nächsthöheren Gang. Dasselbe gilt bei einem gemächlichen Fahrstil, wobei hier eher früher geschalten wird. Lernt das Auto nun von einer Person mit dem zweitgenannten Stil und wird aber auch hin und wieder mit anderen Personen, zum Beispiel Familienmitglieder, geteilt, kann es für diese einen Komfortverlust darstellen. Identifiziert das Auto jedoch durch das Fahrverhalten eine andere Person, könnte es den gelernten Stil temporär vergessen oder gar ein neues Profil anlegen und erneut lernen.

Im Unternehmensbereich, wo Dienstfahrzeuge oder Lieferwägen zum Einsatz kommen, ist es meistens notwendig ein Fahrtenbuch zu führen. Bei einem klassischen Fahrtenbuch werden die gefahrenen Kilometer, Datum und Uhrzeit, Name des Fahrers, Abfahrts- und Zielort sowie die Unterschrift in einem Buch im Fahrzeug festgehalten. Durch das händische Eintragen kann es zu Zeitverlusten, unvollständige Dokumentation und auch manipulierten Daten kommen, was im Endeffekt in hohen Kosten resultiert. Elektronische Fahrtenbücher sind in der Lage diese Daten digital aufzuzeichnen. Mithilfe eines \textit{On-Board-Diagnose} II (OBDII) Dongles werden der Kilometerstand und die Fahrzeugnummer erfasst. Die GPS-Position, Datum und Uhrzeit wie auch die Fahreridentifikation und gegebenenfalls der Zweck der Fahrt werden mit einem gekoppelten Smartphone in das System ergänzt \footnote{Drivebox: \url{https://www.drivebox.at/drivebox_main.html}} \footnote{Vimcar: \url{https://vimcar.de/fahrtenbuch}}. Die Abrechnung und Auswertung lässt sich dadurch erheblich erleichtern, jedoch ist noch immer eine Interaktion des Lenkers erforderlich, was wiederum Raum für Fehler und Manipulation schafft. Kommt eine Fahreridentifikation basieren am Fahrverhalten zum Einsatz, kann das komplette System in das Fahrzeug integriert werden. Die Positionsdaten können dabei von einem integrierten Navigationssystem ausgelesen werden und die restlichen Fahrzeugdaten über den CAN-Bus. Die Fehlerwahrscheinlichkeit verringert sich dabei enorm und die Benutzerfreundlichkeit wird erhöht, da nichts mehr händisch eingetragen werden muss.

Überdies ist auch eine Art Diebstahlwarnung zu realisieren. Dem System sind eine Reihe an Fahrerprofilen bekannt, welche zuvor eingelernt wurden. Bei jeder neuen Fahrt wird überprüft, ob das momentane Fahrverhalten des Lenkers erkannt wird. Stellt das System zum Beispiel zu einer Wahrscheinlichkeit von 90\% fest, dass sich das Profil nicht unten den bekannten befindet, kann beispielsweise eine Benachrichtigung an den Fahrzeughalter versendet, oder die Fahrerin zum Anhalten gebracht werden.

Der Mechanismus kann weiters dazu verwendet werden, Fahrzeug-Funktionen und Leistung fahrerabhängig zu steuern. Ein Familienvater ist so etwa in der Lage, die zur Verfügung stehenden PS einzuschränken, wenn seine Kinder mit dem Auto fahren.

Durch das Aufzeichnen und Analysieren von personenbezogenen Daten kommen natürlich auch datenschutzrechtliche Bedenken auf. Werden die Daten in eine Cloud – sei es eine vom Hersteller, der Versicherung oder einer anderer Drittpartei – gesendet und ausgewertet, können Personen von diesen Unternehmen oder Organisationen eindeutig identifiziert werden. Liegen zudem Standortdaten des Autos vor, kann auch eine genau Ortung durchgeführt werden. Dies kann in einigen Fällen problematisch sein. So könnte eventuell der Hersteller bestimmte Services anbieten, welche personalisierte Werbungen während dem Fahren anzeigen. Zum Beispiel ist es dadurch möglich, bevorzugte Restaurants in der Navigationsansicht hervorzuheben. Auch Versicherung können diese Informationen für sich nutzen, um Fahrer- und Fahrstil abhängige Versicherungspakete anbieten zu können \footnote{Signal Iduna: \url{https://www.app-drive.de}}. Hier wird genauso ein OBDII-Dongle verbunden mit einer App für die Fahrstilanalyse eingesetzt. Werden viele Notbremsungen und rasche Beschleunigen verzeichnet, kann die Versicherungsprämie erhöht werden. Der Dongle und das Smartphone könnten durch das System ersetzt werden.

Die angeführten Beispiele zeigen, dass ein System, welches die Person hinter dem Lenkrad eines Fahrzeuges eindeutig identifiziert, Benutzervorteile bringen kann. Zudem erschließt sich ein neues Geschäftsfeld für Autohersteller, um vielleicht Premium-Features anbieten zu können. Jedoch stellen sich auch Fragen zur Privatsphäre und wie mit solch sensiblen Daten umgegangen wird.

\section{Überblick und Ziele}
\label{sec:overview}

In der Masterarbeit wird ein System vorgestellt, welches den Lenker eines Fahrzeuges anhand des Fahrstils eindeutig identifizieren kann. Ausgangspunkt ist, dass jedes Individuum ein anderes Verhalten in verschiedenen Verkehrssituationen hat. Einer bremst eher früher an dafür nicht so kräftig, währendessen eine später und härter bremst. Person A fährt eine Kurve mit 30 km/h im dritten Gang und lenkt dabei etwas weniger ein. Person B andererseits nimmt eine Kurve schneller und muss daher mehr einlenken. Dafür muss vielleicht die Lenkradposition in der Kurve nicht mehr angepasst werden, wohingegen vielleicht einmal kurz die Bremse gedrückt wird. All diese Informationen - Geschwindigkeit, Drehmoment, Lenkradwinkel, Bremsdruck usw. - werden über den \textit{Controller-Area-Network} (CAN) Bus als Nachrichten ausgetauscht. \textit{Electronic-Control-Units} (ECUs) verarbeiten diese und steuern dementsprechend den Motor, das Getriebe oder eine Öldruckpumpe an.

Um aus vershiedenen CAN-Daten einen Zusammenhang zum Lenker herstellen zu können, wird eine Untermenge von \textit{Artificial Intelligence} (AI), nämlich \textit{Machine Learning} (ML) verwendet. Bei ML kommen mehrere mathematische Algorithmen zum Einsatz, welche statistische Modelle aufgrund von Trainingsdaten aufbauen. Die Modelle versuchen daraufhin ein Muster in den Daten zu erkennen, um später eine Aussage über den Lenker treffen zu können. Die Lerndaten bestehen dabei aus den verschiedenen CAN-Nachrichten und einer Fahreridentifikation - dem Zielwert. Nach der Trainingsphase des Modells werden nur noch die CAN-Daten eingespielt. Als Resultat wird die Fahrer-ID mit größen zutreffenden Wahrscheinlichkeit ausgegeben \cite{Conway2012}.

Das Ziel der Masterarbeit ist bereits existierende Methoden zur Fahreridentifizierung mit \textit{Machine Learning} (siehe \ref{sec:related_work}) und den vorliegenden Daten (siehe \ref{sec:data_description}) zu validieren beziehungsweise zu optimieren. Im Zuge dessen soll herausgefunden werden, wie lange die Trainingsphase eines ML-Modells sein muss, um eine Trefferquote von über 85\% erzielen zu können. Des Weiteren gilt es die Dauer zu bestimmen, welche für die Identifikation (zu 85\%) eines Lenkers mit einem bereits trainierten Modell benötigt wird. Ein weiteres Ziel ist das System in ein Fahrzeug zu integrieren und mit (fast) Echtzeit-Daten zu erproben. Hierfür wird ein Embedded-Device mit beschränkten Ressourcen eingesetzt.

\section{Struktur der Arbeit}
\label{sec:structure}

% Needs work

Diese Arbeit geht eingangs auf die Motivation und Problemstellung ein. Danach wird ein Überblick mit den Zielen geschaffen, sowie vergleichbare Arbeiten vorgestellt. Im zweiten Kapitel werden die Grundlagen behandelt. Das beinhaltet vorwiegend den CAN-Bus und \textit{Machine Learning}. Kapitel 3 beschreibt die Umsetzung des Systems. Dabei wird auf die vorliegenden Daten eingegangen und wie diese bestmöglich vorbereitet werden. Des Weiteren wird auch die Implementierung verschiedener ML-Algorithmen beschrieben und wie damit der Lenker eins Fahrzeuges identifiziert werden kann.

\section{Verwandte Arbeiten}
\label{sec:related_work}

Zu diesem Thema sind bereits einige Papers zu finden. Zum Beispiel konnten 2005 Forscher aus Japan eine Fahreridentifizierung mit einer Genauigkeit von 73\% durchführen \cite{Wakita2005}. Sie verwendeten jedoch CAN-Bus Nachrichten von einem Simulator. Später wurde die Trefferquote im Labor zwar auf 89.6\% erhöht aber die Anwendung unter realen Bedingungen brachten nur 71\% ein \cite{Miyajima2007}. Im folgenden werden noch drei Papers vorgestellt, welche die Identifikation ausschließlich mit echten Fahrzeugdaten untersucht haben.

In einem Paper von 2016 wurde untersucht, ob Einzelpersonen basierend auf ihren natürlichen Fahrverhalten identifiziert werden können. Für die Datenbasis wurden CAN-Nachrichten eines Serienfahrzeuges verwendet. 15 Teilnehmerinnen mussten zuerst bestimme Manöver auf einem abgesperrten Parkplatz durchführen und danach eine ca. 80 km lange vordefinierte Strecke abfahren. Für die Analyse wurde \textit{Machine Learning} mit verschiedenen Algorithmen verwendet. Dabei konnte festgestellt werden, dass bei einem 1 zu 1 Vergleich die Teilnehmer zu 100\% unterscheidbar sind. Des Weiteren konnte eine hohe Identifikationswahrscheinlichkeit bei nur acht Minuten Trainingszeit erzielt werden \cite{Enev2016}.

Die Arbeit von B. Gahr et. al. von 2018 setzt auf die soeben beschriebene auf. Da gezeigt wurde, dass eine Identifikation zu 100\% möglich ist, hat diese es versucht, die Methoden mit anderen Daten zu validieren. Dafür wurden aber nicht direkt die Nachrichten vom CAN-Bus abgegriffen, sondern über ein Smartphone, welches über Bluetooth mit einem OBDII-Dongle verbunden wurde. Hierbei konnte mit den Methoden jedoch nur eine Identifikationsrate von maximal 70\% erzielt werden. Daher wurde ein anderer Ansatz gewählt, bei dem nur Daten während eines Bremsvorgangs in Betracht gezogen werden. Das hat zu Ergebnissen zwischen 80 und 99,5\% geführt \cite{Gahr2018}.

Ein anderes Paper verfolgte einen ähnlichen Ansatz, bei dem nur die Daten während einer Kurve berücksichtigt werden. Die CAN-Nachrichten wurden hierbei mit einem proprietären Data-Logger aufgezeichnet. Es folgte eine Analyse der 12 häufigsten Kurven im Datensatz. Dabei konnte ein Fahrer verglichen mit einem zweiten Fahrer mit einer Wahrscheinlichkeit von fast 77\% unterschieden werden. Besteht das Set aus fünf Fahrern, liegt die Identifikationsrate bei 50,1\% \cite{Hallac2016}.

\section{Neuigkeitswert}
\label{sec:novelty}

Die Masterarbeit wird teilweise auf die bereits bestehenden Papers aufsetzten und bewehrte Methoden übernehmen. Ein wesentlicher Unterschied ist aber, dass die hier vorliegenden Daten (siehe \ref{sec:data_description}) weder in einem bestimmten Setting noch unter anderen Kontrolleinflüssen und direkt vom CAN-Bus mitgemessen worden sind. Wie auch aus den Zielen hervorgeht, wird versucht, die Methoden dahingehend zu verbessern, sodass eine möglichst schnelle Identifikation durchgeführt werden kann.