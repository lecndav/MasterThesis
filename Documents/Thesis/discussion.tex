%----------------------------------------------------------------
%
%  File    :  discussion.tex
%
%  Authors :  David Lechner, FH Campus Wien, Austria
%
%  Created :  10 Oct 2019
%
%  Changed :  10 Oct 2019
%
%----------------------------------------------------------------


\chapter{Diskussion}
\label{chap:discussion}

Das letzte Kapitel vor der Zusammenfassung präsentiert die erzielten Ergebnisse der Masterarbeit und diskutiert die Erkenntnisse. Die Gliederung spiegelt die behandelten Forschungsfragen. Des Weiteren wird ein Ausblick über weiterführende Tätigkeiten gegeben.

\section{Validierung der Methoden}

Im Abschnitt \ref{sec:related_work} wurden vier Paper vorgestellt, welche sich mit dem gleichen Thema - Fahreridentifikation basierend am Fahrverhalten - beschäftigen. Diese haben hauptsächlich den \textit{Machine Learning} Algorithmus \textit{Random Forest} verwendet und damit gute Resultate erzielen können (teilweise über 85\%). Es wurde jedoch sehr wenig über die Struktur, Eigenschaften und Herkunft der Daten erwähnt. Daher war die erste Frage: Sind die existierenden Methoden mit den vorhandenen Daten valide? Kapitel \ref{chap:set_up} hat gezeigt, dass dies zutrifft, da bei einem ersten Versuch eine Genauigkeit des Modells von 85\% herausgekommen ist. Weitere Durchläufe haben hingegen eine hohe Schwankungsbreite - von 78,66\% bis 92,53\% - ergeben. Das lässt sich auf zwei Gründe zurückführen. In Sektion \ref{sec:feature_optimization} bei der \textit{Feature}-Analyse wurde entdeckt, dass das Signal \textit{Gangposition} Unregelmäßigkeiten aufweist. Bei fünf von neun Datensets ist der Wert durchgehend \textit{14}. Die guten Ergebnisse (über 90\%) können zum Teil daher kommen, dass das Trainings- und Testset gleichverteilt Datenpunkte mit und ohne Gangposition \textit{14} haben. Dadurch bekommt, wie in Abbildung \ref{fig:feature_importance} gezeigt, das \textit{Feature} eine große Bedeutung und hilft ungemein bei der Klassifizierung. Der zweite Grund ist, dass die Auswahl der Daten einen Einfluss auf die Genauigkeit haben. Befinden sich viele Autobahnkilometer, welche wenig Individualität aufweisen und schwerer zuordenbar sind, unter den Sets, sinkt die Trefferquote. Das Gegenteil tritt bei vielen Datenpunkten von Lenk- oder Bremsmanöver auf, die Trefferquote steigt.

Das \ref{chap:set_up}. Kapitel hat die vorliegenden Daten beschrieben. Im Gegensatz zu \cite{Gahr2018} sind sie direkt von mehreren CAN-Linien abgegriffen worden und nicht über den Diagnostik-Port (OBD II). Das bringt einige Vorteile. Zum Beispiel sind aufgrund dessen alle vorhandenen Signale am Bus verfügbar. Alleine am Fahrwerks-CAN sind bis zu 1500 und am Komfort-CAN über 2500 Signale vorhanden. Je nach Hardwarekapazitäten und Fahrzeugmodell können auch noch mehrere CAN-Linien miteinbezogen werden. Damit wird das ML-Modell genauer und aussagekräftiger, wenn es mehr Daten zum Verarbeiten gibt. Währenddessen über den OBD II Port \cite{SAEII} laut Spezifikation maximal 196 Signale übertragen werden. Hier handelt es sich meist auch nur um verschiedenen Abgaswerte, Fehlerhistorie und Temperaturen. Hersteller können zudem noch weitere Signale in die Schnittstelle implementieren, welche aber nicht spezifiziert sind. Um die Werte über OBD II auslesen zu können, muss ein Dongle eingesetzt werden, der in einen dezidierten Anschluss, meist versetzt unterhalb des Lenkrads, eingesteckt wird. Er hat das OBD Protokoll implementiert und kann die Signale auswerten und in die physikalische Einheit umrechnen \cite{OBD20}. Für die Weiterverarbeitung wird gewöhnlich eine Bluetooth-Verbindung zu einem weiteren Gerät (z.B. Smartphone) benötigt. Für diese Anwendung wäre die ALEN notwendig, die die Daten empfängt und mit dem \textit{Machine Learning} Algorithmus klassifiziert. Durch den OBD II Dongle ergeben sich zusätzlich einige Angriffsvektoren, mithilfe dieser die Privatsphäre verletzt, Eigentum beschädigt und sogar Leib und Leben gefährdet werden könnte. Forscher haben hierzu 77 am Markt verfügbare Dongles getestet und sind zum Entschluss gekommen, dass alle mindestens zwei Schwachstellen enthalten \cite{USENIX20}. Beispielsweise sind keine Authentifizierungsmethoden realisiert, lassen das Extrahieren der Firmware ohne Sicherheitsmechanismus zu oder senden CAN-Nachrichten am Bus ohne jegliche Validierung. Das Abgreifen direkt vom Bus ist daher die klar bessere Methode und bietet mehr Signale, bessere Sicherheit und Schutz der Privatsphäre.

\section{Optimierung}

Bei der Beantwortung der zweiten Forschungsfrage wurde versucht, den \textit{Random Forest} Algorithmus zu optimieren. Im ersten Schritt sind die Parameter analysiert worden. Dabei hat sich herausgestellt, dass die Standardwerte bereits sehr gute Ergebnisse liefern. Selbst nach 16128 Durchläufen, bei denen jede mögliche Kombination der Parameter überprüft wurde, konnte lediglich eine Verbesserung um fast ein Prozent erzielt werden. Auffallend war, dass eine Veränderung des Wertes bei dem Parameter \textit{criterion} - von \textit{gini} auf \textit{entropy} - zwar keine Auswirkung auf die Trefferquote gehabt hat, jedoch sich die Durchlaufzeit erheblich vergrößert hat, nämlich um mehr als 300\%. Der Grund hierfür ist die dahinterstehende Formel. \textit{Entropy} wird mit dem Logarithmus der Basis 2 von der Wahrscheinlichkeit berechnet, wohingegen bei der \textit{Gini Impurity} die Wahrscheinlichkeit mit sich selbst multipliziert wird (siehe \ref{sec:random_forest}). Da die Logarithmus-Funktion mehr Rechenleistung erfordert als eine Multiplikation, ist das Aufteilen eines Knotens mit \textit{entropy} langsamer.

Durch eine detailliertere Analyse der vorhanden Datensätze hat sich eines verdeutlicht: Die Daten sind das ausschlaggebendste für die \textit{Machine Learning} Anwendung. Diese Aussage wird auch immer wieder in der Literatur (\cite{Rebala2019}, \cite{Conway2012}) bestätigt. Damit ist gemeint, dass zum einen die Struktur einheitlich und bekannt und zum anderen ausreichend viele Daten erhältlich sein müssen. Gibt es beispielsweise mehrere fehlende Werte oder Unregelmäßigkeiten (siehe \textit{Gangposition}) können manche ML Algorithmen nicht zuverlässig Ergebnisse liefern. Das gleiche gilt wenn einfach zu wenig verfügbar sind. Das Regelset des Modells kann so nicht hinreichend gebildet werden, um die Vielfalt der existierenden Daten abzubilden.

Ein ML-\textit{Model} stellt in gewisser Weise eine \textit{Blackbox} dar, da nicht ohne einer näheren Untersuchung erkennbar ist, auf Grund welcher Daten eine Entscheidung getroffen wird. Diese Erfahrung hat auch der Autor von \cite{Gahr2019} gemacht, der ebenfalls den Lenker anhand von CAN- und Smartphone-Daten identifizieren wollte. Er hat das Modell mit allen zugänglichen Daten trainiert, darunter auch Längen- und Breitengrade des GPS-Moduls. Der \textit{Classifier} hat am Ende nicht basierend am Fahrverhalten Datenpunkte zugeordnet, sondern alleine mittels den GPS-Daten. In Abschnitt \ref{sec:feature_importance} wurde ähnliches aufgezeigt. Mit Hilfe der integrierten \textit{Feature Importance} des \textit{Random Forest}s ist bemerkt worden, dass es beim Signal \textit{Gangposition} keine durchgängigen Werte gibt. Der Missstand wurde bereits oben erläutert und bekräftigt die genannte These, dass die Daten vor dem Einsatz mit \textit{Machine Learning} genauestens analysiert und validiert werden müssen. Das verhindert eine falsche Vorgehensweise des ML-Modells als \textit{Blackbox}.

\section{Dauer der Trainings- und Identifikationsphase}

Im Zuge der Optimierungen und der Integration wurde auch die Fragestellung behandelt, wie viele Trainingsdaten vonnöten sind, um mindestens 85\% Genauigkeit des \textit{Models} zu erreichen. Des Weiteren war die Frage, wie lang eine Person fahren muss, damit sie zu 85\% identifiziert werden kann. Hier kann keine pauschale Antwort gegeben werden, da viele Faktoren Einfluss darauf nehmen. Mitsicherheit die maßgeblichsten davon sind die Anzahl an Fahrprofilen in einem Set und welche Daten ausgewählt werden. Die Abbildungen \ref{fig:train_duration_shuffle} und \ref{fig:train_duration_start} zeigen den Unterschied zwischen einer zufälligen und chronologischen Auswahl. Bei ersteren ist die maximale Genauigkeit mit 40 Minuten Trainingszeit pro Fahrer bei etwa 75\% gelegen. Sind die Daten jeweils nach Fahrtbeginn für die Trainingsphase herangezogen worden, wurde nach ca. 15 Minuten knapp 86\% erreicht. Mit der Zufallsauswahl konnte das Ziel somit nicht erreicht werden. Das bestätigt die Annahme, dass Daten am Beginn einer Fahrt, wo mehrere Lenk- und Bremsmanöver stattfinden, eine höhere Individualität aufweisen und dadurch besser zu klassifizieren sind. Die beschriebenen Tests sind mit neun Fahrprofilen durchgeführt worden. Während der Integration hat ein Test mit nur vier eine Genauigkeit von 94\% ergeben. Das beweist, dass allein die Reduktion auf eine geringere Anzahl einen großen Einfluss hat.

Für die Dauer der Identifikationsphase gilt ebenso das gleiche. Da sich das Kapitel \textit{Optimierung} an erster Stelle auf die allgemeine Verbesserung des Modells konzentriert hat, ist dieser Aspekt nicht beleuchtet worden. Es liegen daher keine Ergebnisse für ein Modell mit neun Profilen vor. Das vorherige Kapitel ist jedoch explizit auf die Fragestellung eingegangen. Allerdings kann wieder keine pauschale Aussage getroffen werden, da ein paar Gegebenheiten vorausgesetzt waren. Zum einen sind im ML-Modell nur vier Datensätze inkludiert und zum anderen gibt es eine Einschränkung durch die Zeit die benötigt wird, um den \textit{Random Forest Classifier} zu trainieren. Des Weiteren ist die Art der Identifizierung durch den eingesetzten Algorithmus (siehe \ref{sec:identification_difference}) gesteuert. So muss zum Beispiel fünfmal ($\widehat{=}$ 5 Minuten) hintereinander die gleiche Fahrerin identifiziert werden. Mehreren Tests mit verschiedenen Daten haben hierbei ergeben, dass 97\% davon im Durchschnitt nach 5,73 Minuten identifiziert sind.

\section{Integration}

Die letzte Forschungsfrage zielt auf die Integration des Systems zur Fahreridentifikation in ein Fahrzeug ab. Wie in dem Kapitel beschrieben, ist es nicht möglich gewesen, da ein tatsächliches Auto nicht zur Verfügung gestellt wurde. Stattdessen ist der CAN-Bus durch die bereits vorhandenen Daten simuliert worden. Die Beschreibungen und Ergebnisse haben gezeigt, dass dies sehr gut funktioniert hat und es auf diese Weise zu fast realen Bedingungen gekommen ist. Die ALEN als zentrale Einheit eignet sich prinzipiell gut für diesen Einsatz, da sie die notwendige Peripherie mitbringt, hat aber klar ihre Grenzen in Bezug auf \textit{Machine Learning}. So dauert es ca. fünf Minuten, bis die CCU den \textit{Random Forest Classifier} erstellt und trainiert hat. Einige Maßnahmen könnten die Zeit reduzieren. Beispielsweise kann die ALEN durch eine leistungsstärkere CCU mit sonst ähnlicher Hardwareausstattung ersetzt werden. Denkbar wäre zudem, das trainierte Modell zu speichern und bei Fahrtbeginn nur zu laden. Den Prozess komplett in die \textit{Cloud} auszulagern, erfordert das Umsetzen von weitgehenden \textit{Security}- und \textit{Privacy}-Anforderungen, da sich die persönlichen Fahrprofile nicht mehr nur lokal im Auto befinden. Weiters würde dadurch das Konzept von \textit{Edge Computing} verloren gehen. Die Vorteile davon wurden in den Grundlagen näher gebracht, hat aber eben auch einen Nachteil, nämlich der von fehlenden Ressourcen am \textit{Edge Device}, vorausgesehen.

Selbst wenn die Zeit auf ein Minimum reduziert werden könnte, würde es die Identifizierung nicht zwangsläufig beschleunigen. Sie hängt hauptsächlich vom Zweck ab und wird von der Konfiguration des \textit{ML-Model Predictor} bestimmt. Dem Serviceanbieter - sei es der OEM oder eine Drittpartei - muss klar sein, dass je kürzer die Identifikationsphase gewählt wird, desto leichter kann es zu einer Missidentifikation kommen. Es gibt durchaus Services, welche erst gegen Ende einer Fahrt die Identifikation benötigen. Zum Beispiel ist in der Einleitung das digitale Fahrtbuch besprochen worden, das vollkommen ohne jeglicher Interaktion seitens der Fahrerin auskommt. Eine Implementierung auf der ALEN wäre ohne viel Zutun möglich, da dort bereits alle Daten aufliegen. Eine weitere Einsatzmöglichkeit liegt bei \textit{Car-Sharing} Anbietern. Sie können mit Hilfe des Systems erkennen, ob wirklich die registrierte Person hinter dem Lenkrad sitzt. Diese Information genügt es auch erst am Ende zu haben. Das gleiche Prinzip lässt sich zudem bei Mietwagenunternehmen realisieren. Die Standardverträge erlauben nur einen Lenker, jede weitere kostet extra Geld (\EUR{8,50} bei \textit{Sixt} \cite{SIXT}). Die Überprüfung kann hiermit durchgeführt und bei Bedarf der Vertragszusatz automatisiert ergänzt werden.

Es gibt überdies Anwendungen, die eine schnelle Identifikation erfordern. Eine davon, die Diebstahlwarnung, wurde ebenfalls schon in der Einleitung erwähnt und in diesen Kontext gesetzt. Für die Umsetzung wäre bloß eine kleine Adaption der vorhandenen Anwendung notwendig, bei der eine zusätzliche Abfrage zu unzureichenden Klassifizierungen eingebaut wird. Nachdem mehrere Versuche Dateien korrekt zuzuordnen gescheitert sind, kann eine SMS an den Fahrzeughalter gesendet werden, die darüber informiert, dass womöglich eine nicht autorisierte Person das Fahrzeug steuert. Hier muss aber klar geprüft werden, ab wann entschieden wird, dass es sich wirklich um einen unbekannten Fahrer handelt und nicht um einen bekannten, der noch nicht identifiziert wurde.

Durch die Ergebnisse des letzten Kapitels ist davon auszugehen, dass ein paar Minuten an Fahrzeit benötigt werden, damit eine Lenkerin ausreichend erkannt wird. Aus diesem Grund sind einige Services nicht umsetzbar. Allen voran jene, welche unmittelbar nach Fahrtbeginn in Kraft treten würden. Das automatische Einstellen des bevorzugten Radiosenders oder das Einrichten der Außenspiegel fallen somit weg. Ebenso die Möglichkeit, dass Fahrerabhängig bestimmte \textit{Features} de- und aktiviert werden, ist nicht vorstellbar. Das gleiche gilt für eine Drosselung der Leistung, wenn beispielsweise noch unerfahrene Personen hinter dem Steuer sitzen. Dies könnte gegebenenfalls sogar eine Gefahr darstellen, da in der Anfangszeit die volle Leistung zu einem erhöhten Unfallrisiko beiträgt.

\section{Limitierungen}

Da nur eine geringe Anzahl an Fahrdaten vorliegen, kann keine Aussage über die Genauigkeit des Systems in einer diverseren Umgebung erfolgen. Zudem fehlen CAN-Nachrichten von mehr Fahrzeugtypen, um das Modell mit den Signalen validieren zu können. Des Weiteren sind keine Informationen über Geschlecht, Altersgruppen und Dauer des Führerscheinbesitzes von den vorhandenen Fahrprofilen bekannt. Es ist daher auch unklar, ob es sich dabei um eine repräsentative Gruppe an Testpersonen handelt. Bei nur neun ist es anzunehmen. Obwohl die Simulation den realen Bedingungen sehr nahe kommt, können trotzdem nicht alle Eventualitäten abgebildet werden. Das resultiert wahrscheinlich in einer Abweichung der Zuverlässigkeit der Anwendung.

\section{Ausblick}

Um das vorgestellte System zur Fahreridentifikation zu verbessern, gibt es mehrere Möglichkeiten. Neben dem \textit{Random Forest} sind in der Literatur auch andere Algorithmen zur Klassifizierung (\cite{Rebala2019}, \cite{Enev2016}) beschrieben. Bei einem Vergleich könnte sich ein anderer als performanter und genauer herausstellen. Das Modell kann zusätzlich optimiert werden, indem eine \textit{Feedback}-Schleife implementiert wird. Nach einer erfolgreichen Identifikation werden die neuen Messpunkte dem Modell als neue Trainingsdaten zur Verfügung gestellt. Dadurch lernt der \textit{Classifier} durchgehend neue Fahrsituationen. Ein anderer Ansatz wäre nur Datenpunkte herzunehmen, welche leichter zu klassifizieren sind. Kapitel \ref{chap:set_up} und \ref{chap:optimization} haben aufgezeigt, dass solche existieren. Hier stellt sich die Frage, welche Fahrmanöver eine höhere Individualität aufweisen. Die Messdateien können folglich im \textit{Data-Preprocessor} auf die relevanten Daten reduziert werden. Als Anhaltspunkt dienen hierfür die Arbeiten \cite{Gahr2018} und \cite{Hallac2016}, die sich zum Teil schon damit beschäftigt haben.

Ein weiterer Punkt ist die Anwendung mit einem echten Fahrzeug zu erproben, da dies nicht möglich war. Im Zuge dessen ist eine Überprüfung der Simulationsergebnisse möglich. Außerdem wäre es interessant zu wissen, ob sich das System durch eine manipulierte Fahrweise austricksen lässt. Dadurch kann die Stabilität getestet werden und ob die Fahrweise von einer Person nachgeahmt werden kann.